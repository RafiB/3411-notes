\part{Week 6}
\chapter{Logic}
\section{Logical Agents}
Humans know things, and what we know helps us do things.

We have knowledge-based agents, which use internal representations of knowledge
and reason about this knowledge to infer what to do.

We also have logical agents, which use logical sentences to represent
knowledge, and reason by inferring new sentences.

\subsection{Example: General Game-Playing Agents}
General Game Players are systems that are able to understand formal
descriptions of arbitrary games, and are able to learn to play these games
effectively. They don't know the rules of a game until they start playing.

Unlike specialised game players (like Deep Blue (chess) or Chinook (checkers))
they do not use algorithms in advance for specific games.

\subsubsection{How to represent game rules}
It is possible in principle to communicate game information in the form of
tables (for legal moves, state update)

\paragraph{Problem} the size of the description - even if state is finite, the
necessary tables can be large (for example, there are $10^{44}$ states in
Chess)

\paragraph{Solution} in many cases, worlds are best thought of in terms of
atomic features tha may change (``position of white queen'', ``black can
castle'', etc). We can represent features directly and use logic to describe
how actions change individual features rather than entire states.

\subsubsection{Game Description Language (GDL): Facts and Rules}
In the following, bold/blue symbols are pre-defined keywords in the logic-based
GDL.

\textbf{Facts}
\begin{lstlisting}[morekeywords={role, init}]
role(xplayer)
role(oplayer)

init(cell(1,1,b))
init(cell(1,2,b))
...
init(cell(3,3,b))

init(control(xplayer))
\end{lstlisting}

\textbf{Rules}
\begin{lstlisting}[morekeywords={legal, true, next, does}]
legal(W,mark(M,N)) <=
    true(cell(M,N,b)) $\land$ true(control(W))

next(cell(M,N,x)) <=
    does(xplayer,mark(M,N))

next(cell(M,N,o)) <=
    does(oplayer,mark(M,N))
\end{lstlisting}

\subsection{Example: Wumpus World}
\subsubsection{Agent's actuators}
\begin{itemize}
    \item turn $90^{\circ}$ left or right
    \item move forward one square
    \item grab the gold
    \item shoot an arrow in the current direction
    \item climb out of the cave from [1,1]
\end{itemize}

\subsubsection{Agent's percepts}
\begin{itemize}
    \item a stench when adjacent to the wumpus
    \item a breeze when adjacent to a pit
    \item a glitter when in the square with gold
    \item a bump when walking into a wall
    \item a scream when the wumpus is killed
\end{itemize}

\subsubsection{Agents in the Wumpus World}
The main challenge is the initial ignorance of the configuration of the
environment. The solution to this problem is to draw the right inferences from
the available information.

\subsection{Propositional Logic}
\subsubsection{Vocabulary}
\begin{tabular}{l l l}
Proposition symbols: & p, q, r\\\\
Logical connectives: & $\neg{p}$             & (negation, ``not p'')\\
                     & $p\land{q}$           & (conjunction, ``p and q'')\\
                     & $p\lor{q}$            & (disjunction, ``p or q'')\\
                     & $p\Rightarrow{q}$     & (implication, ``p implies q'')\\
                     & $p\Leftrightarrow{q}$ & (equivalence, ``p if and only if q'')\\\\
Sentences:           & \multicolumn{2}{p{8cm}}{built from proposition symbols and connectives, e.g.
$ready \Leftrightarrow (a4paper \lor a3paper) \land \neg{jam}$}\\\\
Order precedence:    & $\neg, \land, \lor, \Rightarrow, \Leftrightarrow$\\
\end{tabular}

\subsubsection{Example: Knights and Knaves}
\paragraph{Assumptions} knights always tell the truth, knaves always lie
\paragraph{Example} Two people, A and B. A says ``One of us is a knave!''

Truth table:\\
\begin{tabular}{l l | l l l l l}
\hline
p     & q     & $\neg{p}$ & $p \land q$ & $p \lor q$ & $p \Rightarrow q$ & $p \Leftrightarrow q$\\ \hline
false & false & true      & false       & false      & true              & true\\ \hline
false & true  & true      & false       & truee      & true              & false\\ \hline
true  & false & false     & false       & true       & false             & false\\ \hline
true  & true  & false     & true        & true       & true              & true\\ \hline
\end{tabular}\\

Example:\\
\begin{tabular}{l l | l l l l}
\hline
A     & B     & $\neg A$ & $\neg B$ & $\neg A \lor \neg B$ & $A \Leftrightarrow \neg A \lor \neg B$\\ \hline
false & false & true     & true     & true                 & false\\ \hline
false & true  & true     & false    & true                 & false\\ \hline
true  & false & false    & true     & true                 & true\\ \hline
true  & true  & false    & false    & false                & false\\ \hline
\end{tabular}

\subsubsection{Models}
A \textbf{model} $\mathcal{M}$ fixes the truth value (true of false) for every
proposition symbol. For example, $\mathcal{M} = \{A=true, B=false\}$.\\\\
$\mathcal{M}$ satisfies a proposition $p$ iff $p = \text{true} \in \mathcal{M}$\\
$\mathcal{M}$ satisfies a sentence $\neg \alpha$ iff $\mathcal{M}$ does not satisfy
$a$.\\
$\mathcal{M}$ satisfies a sentence $\alpha \land \beta$ iff $\mathcal{M}$
satisfies $\alpha$ and $\mathcal{M}$ satisfies $\beta$\\
$\mathcal{M}$ satisfies a sentence $\alpha \lor \beta$ iff $\mathcal{M}$
satisfies $\alpha$ or $\mathcal{M}$ satisfies $\beta$ or both.\\
$\mathcal{M}$ satisfies a sentence $\alpha \Rightarrow \beta$ iff $\mathcal{M}$
satisfies $\beta$ whenever $\mathcal{M}$ satisfies $\alpha$.\\
$\mathcal{M}$ satisfies a sentence $\alpha \Leftrightarrow \beta$ iff
$\mathcal{M}$ satisfies $\alpha$ if and only if $\mathcal{M}$ satisfies $\beta$\\\\
Example: $\mathcal{M} = \{A = \text{true}, B = \text{false}\}$ satisfies $A
\Leftrightarrow \neg A \lor \neg B$

\subsubsection{Logical Reasoning}
$\beta$ \textbf{follows logically} from $\alpha$, written $\alpha \models
\beta$ iff every model that satisfies $\alpha$ also satisfies $\beta$. $A \land
\neg B$ follows logically from $A \Leftrightarrow \neg A \lor \neg B$.

\paragraph{Example 1} We have one person, ``A''. A says ``I am a knight''.
\subparagraph{How many models satisfy this sentence?} 2: {A = true} and {A =
false}
\subparagraph{What follows logically from this sentence?} $A \lor \neg A$

\paragraph{Example 2} We have one person, ``B''. B says ``I am a knave.''
\subparagraph{How many models satisfy this sentence?} There are no models as
the sentence is contradictory. TODO is this right?

\subsubsection{Logical Reasoning in the Wumpus World}
\begin{tabular}{l l}
$P_{x,y}$ & there is a pit in [x,y]\\
$W_{x,y}$ & there is a wumpus in [x,y]\\
$B_{x,y}$ & the agent perceives a breeze in [x,y]\\
$S_{x,y}$ & the agent perceives a stench in [x,y]
\end{tabular}

TODO add image from lecture notes Logic slide 20

\begin{itemize}
    \item There is no pit in [1,1]:\\
          $\neg P_{1,1}$
    \item A breeze in [1,1] means there is a pit in [1,2] or [2,1]:\\
          $B_{1,1} \Leftrightarrow P_{1,2} \lor P_{2,1}$
    \item A breeze in [2,1] means there is a pit in [1,1], [2,2] or [3,1]:\\
          $B_{2,1} \Leftrightarrow P_{1,1} \lor P_{2,2} \lor P_{3,1}$
    \item The breeze percepts for the first two squares visited are:\\
          $\neg B_{1,1}$ and $B_{2,1}$
\end{itemize}

The conjunction KB of the above 4 sentences is


\begin{align*}
    KB&: (\neg P_{1,1}) \land (B_{1,1} \Leftrightarrow P_{1,2} \lor
          P_{2,1}) \land\\
      &  \indent (B_{2,1} \Leftrightarrow P_{1,1} \lor P_{2,2} \lor P_{3,1}) \land
          (\neg B_{1,1} \text{ and } B_{2,1})\\
      &: (\neg P_{1,1}) \land (\neg (P_{1,2} \lor P_{2,1})) \land (P_{1,1}
         \lor P_{2,2} \lor P_{3,1}) & \text{ because } \neg B_{1,1} \text{ and }
         B_{2,1}\\
      &: (\neg P_{1,2} \land \neg P_{2,1}) \land (\neg P_{1,1} \land P_{1,1}
         \lor P_{2,2} \lor P_{3,1}) & \text{ by de Morgan's}\\
      &: (\neg P_{1,2} \land \neg P_{2,1}) \land (P_{2,2} \lor P_{3,1})
\end{align*}

Thus,
\begin{align*}
    KB \models&(\neg P_{1,2} \land \neg P_{2,1}) \land (P_{2,2} \lor P_{3,1})\\
    \text{or}\\
    KB \models&\neg P_{2,1} \land (P_{2,2} \lor P_{3,1})
\end{align*}

\subsection{First-Order Logic}
\subsubsection{Vocabulary}
\begin{tabular}{l l}
    Quantifiers         & $\forall, \exists$\\
    Constants           & e.g. \texttt{richard, john}\\
    Variables           & e.g. \texttt{X, Y, Z}\\
    Predicates          & e.g. \texttt{king, married, loves}\\
    Functions           & e.g. \texttt{father, mother}\\
    Logical connectives & $\neg, \land, \lor, \Rightarrow, \Leftrightarrow$\\
\end{tabular}

The \textbf{arity} of a function or predicate is the number of arguments that
can be supplied.

A \textbf{term} can be a variable (X), a constant (richard), or a functional
term (mother(father(john)).

A \textbf{sentence} can be an atomic structure, e.g.
\texttt{married(father(richard), mother(john))}, or a complex sentence, e.g.
\texttt{$\forall$X $\exists$ Y loves(X,Y) $\Rightarrow$ $\exists$Y $\forall$X
loves(X,Y)}

\subsubsection{Other examples of First-Order Logic Sentences}
TODO insert image from Logic slide 23

\subsubsection{Semantics}
The \textbf{Herbrand base} is the set of all atomic sentences without
variables, e.g. \verb|{king(richard), king(john), married(richard,mother(john)), ...}|.

In first-order logic, a \textbf{model} $\mathcal{M}$ fixes the truth value
(true or false) for every element in the Herbrand base. From the above example,
the model would be \verb|{king(richard)=true, king(john)=false, ...}|.

\subsubsection{Logic Entailment}
$\mathcal{M}$ satisfies a variable-free atomic sentence $p$ iff $p = \text{true} \in \mathcal{M}$\\
$\mathcal{M}$ satisfies a sentence $\neg \alpha$ iff $\mathcal{M}$ does not satisfy $a$.\\
$\mathcal{M}$ satisfies a sentence $\alpha \land \beta$ iff $\mathcal{M}$
satisfies $\alpha$ and $\mathcal{M}$ satisfies $\beta$\\
$\mathcal{M}$ satisfies a sentence $\alpha \lor \beta$ iff $\mathcal{M}$
satisfies $\alpha$ or $\mathcal{M}$ satisfies $\beta$ or both.\\
$\mathcal{M}$ satisfies a sentence $\alpha \Rightarrow \beta$ iff $\mathcal{M}$
satisfies $\beta$ whenever $\mathcal{M}$ satisfies $\alpha$.\\
$\mathcal{M}$ satisfies a sentence $\alpha \Leftrightarrow \beta$ iff
$\mathcal{M}$ satisfies $\alpha$ if and only if $\mathcal{M}$ satisfies $\beta$\\\\
$\mathcal{M}$ satisfies a sentence $\forall$X $\alpha$ iff $\mathcal{M}$
satisfies $\alpha$\{X/t\}\footnote{$\alpha$\{X/t\} means replace each occurence of X by t in $\alpha$}
for all variable-free terms t\\
$\mathcal{M}$ satisfies a sentence $\exists$X $\alpha$ iff $\mathcal{M}$
satisfies $\alpha$\{X/t\}\footnotemark[1] for some variable-free term t

We say that $\beta$ \textbf{follows logically} from $\alpha$ ($\alpha \models
\beta$) iff every model that satisfies $\alpha$ also satisfies $\beta$.

\paragraph{Example A} Let $\alpha$ be the sentence \texttt{$\forall$X human(X)
$\Rightarrow$ mortal(X)} (all humans are mortal)\\
Let $\beta$ by the sentence \texttt{human(socrates)} (Socrates is human)


$\mathcal{M}_1 = \texttt{\{human(socrates)=false, mortal(socrates)=false\}}$ satisfies
$\alpha$ because Socrates is not human, and thus does not have to be mortal,
but does not satisfy $\beta$.

$\mathcal{M}_2 = \texttt{\{human(socrates)=true, mortal(socrates)=true\}}$ satisfies both
$\alpha$ and $\beta$.

$\mathcal{M}_2$ is the only possible model that satisfies $\alpha$ and $\beta$,
i.e. $\alpha \land \beta \models mortal(socrates)$.

\paragraph{Example B} Let $\alpha$ be the sentence \texttt{$\exists$X card(X) $\land$
red(X)} (some cards are red)\\
Let $\beta$ be the sentence \texttt{card($\heartsuit 7$) $\land$
card($\clubsuit 8$)} ($\heartsuit 7$ and $\clubsuit 8$ are cards).

\texttt{\{card($\heartsuit 7$)=true, card($\clubsuit 8$)=true, red($\heartsuit 7$)=true,
red($\clubsuit 8$)=false\}} satisfies $\alpha \land \beta$.

However, \texttt{\{card($\heartsuit 7$)=true, card($\clubsuit 8$)=true,
red($\heartsuit 7$)=false, red($\clubsuit 8$)=true\}} also satisfies $\alpha \land \beta$.

Hence, $\alpha \land \beta \not\models \texttt{red($\heartsuit 7$)}$

\section{Applied Logic: Describing Games with GDL}
The GDL uses the following keywords.

\begin{tabular}{ l l }
    role(r)       & r is a role (i.e. a player) in the game\\
    init(f)       & f is true in the initial position or state\\
    true(f)       & f is true in the current state\\
    does(r,a)     & role r does action a in the current state\\
    next(f)       & f is true in the next state\\
    legal(r,a)    & is is legal for r to play a in the current state\\
    goal(r,v)     & r gets goal value v in the current state\\
    terminal      & the current state is a terminal state\\
    distinct(s,t) & terms s and t are syntactically different\\
\end{tabular}

Like Prolog, all variables are implicitly universally quantified.

\begin{lstlisting}[morekeywords={next, does}]
next(cell(M,N,x)) $\Leftarrow$ does(xplayer,mark(M,N))

means $\forall$M,N next(cell(M,N,x)) $\Leftarrow$ does(xplayer,mark(M,N))
\end{lstlisting}

\subsection{Example: Noughts and Crosses}
\subsubsection{Constants}
\begin{tabular}{ l l }
    xplayer, oplayer & Players\\
    x, o, b          & Marks\\
    noop             & Move\\
    0, 50, 100       & Scores\\
\end{tabular}

\subsubsection{Functions}
\begin{tabular}{ l l l }
    \textbf{Name} & \textbf{Arguments}\\
    cell          & (number,number,mark) & Feature\\
    control       & (player)             & Feature\\
    mark          & (number,number)      & Move\\
\end{tabular}

\subsubsection{Predicates}
\begin{tabular}{ l l }
    row      & (number,mark)\\
    column   & (number,mark)\\
    diagonal & (mark)\\
    line     & (mark)\\
    open
\end{tabular}

\subsubsection{Players and initial state}
\begin{lstlisting}[morekeywords={role,init}]
role(xplayer)
role(oplayer)

init(cell(1,1,b))
init(cell(1,2,b))
init(cell(1,3,b))
init(cell(2,1,b))
init(cell(2,2,b))
init(cell(2,3,b))
init(cell(3,1,b))
init(cell(3,2,b))
init(cell(3,3,b))

init(control(xplayer))
\end{lstlisting}

\subsubsection{Move Generator}
Define the following sequence of facts as $\alpha$
\begin{lstlisting}
true(cell(1,1,x))
true(cell(1,2,b))
true(cell(1,3,b))
true(cell(2,1,b))
true(cell(2,2,o))
true(cell(2,3,b))
true(cell(3,1,b))
true(cell(3,2,b))
true(cell(3,3,x))
true(control(oplayer))
\end{lstlisting}

Define the following rules as $\beta$
\begin{lstlisting}[morekeywords={legal,true}]
legal(W,mark(M,N)) $\Leftarrow$
    true(cell(M,N,b)) $\land$
    true(control(W))

legal(xplayer,noop) $\Leftarrow$
    true(control(oplayer))

legal(oplayer,noop) $\Leftarrow$
    true(control(xplayer))
\end{lstlisting}

We then have the following logical consequences:
\begin{align*}
    \alpha \land \beta &\models legal(xplayer,noop)\\
    \alpha \land \beta &\models legal(oplayer,mark(1,2))\\
    ...\\
    \alpha \land \beta &\models legal(oplayer,mark(3,2))\\
\end{align*}

\subsubsection{Game Physics}
At a high level, we can define states in Noughts and Crosses as follows:

If a player marks a cell, that cell belongs to them in the next state ($R_1$)
\begin{lstlisting}[morekeywords={next,does}]
next(cell(M,N,x)) $\Leftarrow$ does(xplayer,mark(M,N))
next(cell(M,N,o)) $\Leftarrow$ does(oplayer,mark(M,N))
\end{lstlisting}

As long as a player doesn't make an illegal move (moving in an occupied cell),
the cells that currently belong to them will still belong to them on the next
state. ($R_2$)
\begin{lstlisting}[morekeywords={next,does,true,distinct}]
next(cell(M,N,W)) $\Leftrightarrow$ does(W,mark(J,K)) $\land$
                     true(cell(M,N,W)) $\land$
                     (distinct(M,J) $\lor$ distinct(N,K))
\end{lstlisting}

On the next state, it will be the other player's turn ($R_3$)
\begin{lstlisting}[morekeywords={next, true}]
next(control(xplayer)) $\Leftarrow$ true(control(oplayer))
next(control(oplayer)) $\Leftarrow$ true(control(xplayer))
\end{lstlisting}

In combination, the above rules have the following logical consequence(s):\\
$R_1 \land R_2 \land R_3 \land \alpha \land does(xplayer, noop) \land
does(oplayer,mark(1,3)) \models next(cell(1,3,o)) \land next(control(xplayer))
\land ...$

\subsubsection{Termination and Goal}
\begin{lstlisting}[morekeywords={terminal,true}]
terminal $\Leftarrow$ line(x)
terminal $\Leftarrow$ line(o)
terminal $\Leftarrow$ $\neg$open

line(X) $\Leftarrow$ row(M,X)
line(X) $\Leftarrow$ column(M,X)
line(X) $\Leftarrow$ diagonal(X)

open $\Leftarrow$ true(cell(M,N,b))
\end{lstlisting}

\begin{lstlisting}[morekeywords={goal}]
goal(xplayer,100) $\Leftarrow$ line(x)
goal(xplayer,50) $\Leftarrow$ $\neg$line(x) $\land$ $\neg$line(o) $\land$ $\neg$open
goal(xplayer,0) $\Leftarrow$ line(o)

goal(oplayer,100) $\Leftarrow$ line(o)
goal(oplayer,50) $\Leftarrow$ $\neg$line(x) $\land$ $\neg$line(o) $\land$ $\neg$open
goal(oplayer,0) $\Leftarrow$ line(x)
\end{lstlisting}

where
\begin{lstlisting}[morekeywords={true}]
row(M,X) $\Leftarrow$ true(cell(M,1,X)) $\land$ true(cell(M,2,X)) $\land$ true(cell(M,3,X))
column(N,X) $\Leftarrow$ true(cell(1,N,X)) $\land$ true(cell(2,N,X)) $\land$ true(cell(3,N,X))
diagonal(X) $\Leftarrow$ true(cell(1,1,X)) $\land$ true(cell(2,2,X)) $\land$ true(cell(3,3,X))
diagonal(X) $\Leftarrow$ true(cell(1,3,X)) $\land$ true(cell(2,2,X)) $\land$ true(cell(3,1,X))
\end{lstlisting}

\section{Knowledge Interchange Format}
Knowledge Interchange Format (KIF) is a standard for programmatic exchange of
knowledge represented in relational logic.

Syntax is prefix version of standard syntax, with some operators renamed (not,
and, or). KIF is case-insensitive. Variables are prefixed with \verb|?|.

\begin{lstlisting}
GDL: r(X,Y) $\Leftarrow$ p(X,Y) $\land$ $\neg$q(Y)
KIF: ($\Leftarrow$ (r ?x ?y) (and (p ?x ?y) (not (q ?y))))
 or: ($\Leftarrow$ (r ?x ?y) (p ?x ?y) (not (q ?y)))
\end{lstlisting}

\subsection{Noughts And Crosses in KIF}
\begin{multicols}{3}
\begin{tiny}
\begin{lstlisting}
(role xplayer)
(role oplayer)
(init (cell 1 1 b))
(init (cell 1 2 b))
(init (cell 1 3 b))
(init (cell 2 1 b))
(init (cell 2 2 b))
(init (cell 2 3 b))
(init (cell 3 1 b))
(init (cell 3 2 b))
(init (cell 3 3 b))
(init (control xplayer))

(<= (next (cell ?m ?n x))
    (does xplayer (mark ?m ?n))
(<= (next (cell ?m ?n o))
    (does oplayer (mark ?m ?n))
(<= (next (cell ?m ?n b))
    (does ?w (mark ?j ?k))
    (true (cell ?m ?n b))
    (or (distinct ?m ?j)
        (distinct ?n ?k)))
(<= (next (control xplayer))
    (true (control oplayer)))
(<= (next (control oplayer))
    (true (control xplayer)))

(<= (legal ?w (mark ?m ?n))
    (true (cell ?m ?n b))
    (true (control ?w)))
(<= (legal xplayer noop)
    (true (control oplayer)))
(<= (legal oplayer noop)
    (true (control xplayer)))

(<= (row ?m ?x)
    (true (cell ?m 1 ?x))
    (true (cell ?m 2 ?x))
    (true (cell ?m 3 ?x)))
(<= (column ?n ?w)
    (true (cell 1 ?n ?x))
    (true (cell 2 ?n ?x))
    (true (cell 3 ?n ?x)))
(<= (diagonal ?x)
    (true (cell 1 1 ?x))
    (true (cell 2 2 ?x))
    (true (cell 3 3 ?x)))
(<= (diagonal ?x)
    (true (cell 1 3 ?x))
    (true (cell 2 2 ?x))
    (true (cell 3 1 ?x)))

(<= (line ?x) (row ?m ?x))
(<= (line ?x) (column ?n ?x))
(<= (line ?x) (diagonal ?x))

<= open
    (true (cell ?m ?n b)))

(<= terminal (line x))
(<= terminal (line o))
(<= terminal (not open))

(<= (goal xplayer 100)
    (line x))
(<= (goal xplayer 50)
    (not (line x))
    (not (line o))
    (not open))
(<= (goal xplayer 0)
    (line o))
(<= (goal oplayer 100)
    (line o))
(<= (goal oplayer 50)
    (not (line x))
    (not (line o))
    (not open))
(<= (goal oplayer 0)
    (line x))
\end{lstlisting}
\end{tiny}
\end{multicols}

\section{Inference}
We will cover inference algorithms for propositional logic, and inference
algorithms for first-order logic.

\subsection{Propositional Inference: Example}
Assumptions: knights always tell the truth, knaves always lie.

We have two people, A and B. A says ``one of us is a knave''.

It follows that A is a knight and B is a knave.

A $\leftrightarrow$ A is a knight
B $\leftrightarrow$ B is a knight

A $\Leftrightarrow$ $\neg$A $\lor$ $\neg$B $\models$ A $\land$ $\neg$B

\subsection{Converting a sentence to CNF}
A \textbf{literal} is an atomic sentence or its negation.\\
A \textbf{clause} is a disjunction of literals.\\
A sentence is \textbf{conjunctive normal form} (CNF) is a conjunction of
clauses.

To convert a propositional sentence into CNF:
\begin{itemize}
    \item eliminate $\Leftrightarrow$ by replacing $\alpha \Leftrightarrow \beta$ with $(\alpha \Rightarrow \beta) \land (\beta \Rightarrow \alpha)$
    \item eliminate $\Rightarrow$ by replacing $\alpha \Rightarrow \beta$ vith $\neg \alpha \lor \beta$
    \item move $\neg$ inwards by repeated application of
        \begin{itemize}
            \item $\neg\neg\alpha \equiv \alpha$ (double negation elimination)
            \item $\neg(\alpha \land \beta) \equiv (\neg\alpha \lor \neg\beta)$ (de Morgan rule)
            \item $\neg(\alpha \lor \beta) \equiv (\neg\alpha \land \neg\beta)$ (de Morgan rule)
        \end{itemize}
    \item Apply distributivity laws whenever possible
        \begin{itemize}
            \item $\alpha \land (\beta \lor \gamma) \equiv ((\alpha \land \beta)\lor(\alpha\land\gamma))$
            \item $\alpha \lor (\beta \land \gamma) \equiv ((\alpha \lor \beta) \land (\alpha \lor \gamma))$
        \end{itemize}
\end{itemize}

\subsection{A Possible Resolution Algorithm}
\textbf{Resolution} uses the principle of proof by contradiction: $\alpha
\models \beta$ iff $\alpha \land \neg \beta$ is unsatisfiable.

\begin{enumerate}
    \item Convert $\alpha \land \neg \beta$ to CNF
    \item Repeatedly apply the resolution rule
        (\url{http://en.wikipedia.org/wiki/Resolution_(logic)#Resolution_in_propositional_logic} that takes two clauses as input
        and produces a new clause
    \item If the resolution rule produces the empty clause we have no conflict,
        and so $\beta$ follows from $\alpha$.
    \item Otherwise, if there are no more new clauses, then $\beta$ does not
        follow from $\alpha$.
\end{enumerate}

\subsubsection{Example}
A says ``one of us in a knave'' Does it follow that A is a knight and B is a
knave?

Prove that A $\Leftrightarrow$ $\neg$A $\lor$ $\neg$B $\models$ A $\land$ $\neg$B

Prove that (A $\Leftrightarrow$ $\neg$A $\lor$ $\neg$B) $\land$ $\neg$(A $\land$ $\neg$B) is unsatisfiable.

Convert to CNF:

\begin{lstlisting}
(A $\Leftrightarrow$ $\neg$A $\lor$ $\neg$B) $\land$ $\neg$(A $\land$ $\neg$B)
((A $\Rightarrow$ $\neg$A $\lor$ $\neg$B) $\land$ ($\neg$A $\lor$ $\neg$B $\Rightarrow$ A)) $\land$ $\neg$(A $\land$ $\neg$B)
(($\neg$A $\lor$ ($\neg$A $\lor$ $\neg$B)) $\land$ ($\neg$($\neg$A $\lor$ $\neg$B) $\lor$ A)) $\land$ $\neg$(A $\land$ $\neg$B)
($\neg$A $\lor$ $\neg$B) $\land$ A $\land$ ($\neg$A $\lor$ B)
$\neg$B $\land$ A $\land$ B
B $\land$ $\neg$B $\land$ A
\end{lstlisting}

which is a contradiction.

Thus we have shown that (A $\Leftrightarrow$ $\neg$A $\lor$ $\neg$B) $\land$
$\neg$(A $\land$ $\neg$B) is unsatisfiable.

And so A $\Leftrightarrow$ $\neg$A $\lor$ $\neg$B $\models$ A $\land$ $\neg$B.

\subsection{First-Order Resolution for Logic Programs/GDL}
Given
\begin{lstlisting}
true(cell(1,1,b))
...
true(cell(3,3,b))
true(control(xplayer))

legal(P,mark(M,N)) <= true(cell(M,N,b)) ∧ true(control(P))
legal(xplayer,noop) <= true(control(oplayer))
legal(oplayer,noop) <= true(control(xplayer))
\end{lstlisting}

answer the query \verb|?- legal(xplayer,M)|

\begin{lstlisting}
(true(control(oplayer)) $\land$ M = noop) $\lor$
    (true(control(xplayer)) $\land$ M = mark(M,N) $\land$ cell(M,N,b))
    $\Rightarrow$ legal(xplayer,M)

(control(xplayer) $\land$ mark(M,N) $\land$ cell(M,N,b))
    $\Rightarrow$ legal(xplayer,M)

control(xplayer) $\land$ (M = mark(M,N) $\land$ cell(M,N,b)
                   $\Rightarrow$ legal(xplayer,M)
\end{lstlisting}

And, since \verb|control(xplayer) = true| we have

\begin{lstlisting}
M = mark(1,1)
M = mark(1,2)
M = mark(1,3)
M = mark(2,1)
..
..
M = mark(3,2)
M = mark(3,3)
\end{lstlisting}

\subsubsection{Substitutions}
To compute logical consequences you need
\begin{enumerate}
    \item unification: using substitutions in order to match two expressions
    \item resolution steps (a single derivation step)
    \item derivations (computing a result for a query)
\end{enumerate}

A \textbf{substitution} is a finite set of replacements of variables by terms,
e.g. \{X/a, Y/f(b), V/W\}

The result of applying a substitution $\Theta$ to an expression $p$ is the
expression SUBST($\Theta$, $p$) obtained from $p$ by replacing every occurrence
of every variable in the substitution by its replacement.

\texttt{SUBST(\{X/a, Y/f(b), V/W\}, p(X,X,Y,Z)) = p(a,a,f(b),Z)}

\subsubsection{Unification}
A substitution $\Theta$ is a \textbf{unifier} for an expression $p$ and an
expression $q$ iff SUBST($\Theta$, $p$) = SUBST($\Theta$, $q$).
\begin{lstlisting}
SUBST(\{M/1,N/3,X/3\},mark(M,N)) = mark(1,3)
SUBST(\{M/1,N/3,X/3\},mark(1,X)) = mark(1,3)
\end{lstlisting}

If two expressions have a unifier, we say they are \textbf{unifiable}.
mark(X,X) and mark(1,3) are not unifiable.

\subsubsection{Most General Unifiers}
A substitution $\Theta$ is \textbf{more general} than a substitution $\theta$
iff $\Theta$ places fewer restrictions than $\theta$ on the values of
variables.

A substitution is a \textbf{most general unifier} (mgu) of two expressions iff
it is more general than any other unifier.

\paragraph{Theorem} If two expressions are unifiable, then they have an mgu
that is unique up to variable permutation.
\begin{lstlisting}
SUBST({M/1,N/X},mark(M,N)) = mark(1,X)
SUBST({M/1,N/X},mark(1,X)) = mark(1,X)

SUBST({M/1,X/N},mark(M,N)) = mark(1,N)
SUBST({M/1,X/N),mark(1,X)) = mark(1,N)
\end{lstlisting}

\subsubsection{Resolution Step for Queries + Logic Program Clauses}
\begin{align*}
    \text{Given:} & \text{Query } L_1 \land L_2 \land ... \land L_m & \text{(without negation)}\\
                  & \text{clauses}                          & \text{(without negation)}\\
\\
    \text{Let:}   & A \Leftarrow B_1 \land ... \land B_n          & \text{``fresh'' variant of clause}\\
                  & \Theta                                  & \text{mgu of $L_1$ and A}\\
\\
    \text{Then:}  & L_1 \land L_2 \land ... \land L_m \rightarrow \text{SUBST($\Theta$,$B_1 \land ... \land B_n \land L_2 \land ... \land L_m$)} & \text{is a \textbf{resolution step}}\\
\end{align*}

\paragraph{Example}
\begin{lstlisting}
legal(P5,mark(M5,N5)) $\Leftarrow$ true(cell(M5,N5,b)) $\land$ true(control(P5))

legal(xplayer,M) $\rightarrow$ with $\Theta$ = {P5/xplayer,M/mark(M5,N5)}

true(cell(M5,N5,b)) $\land$ true(control(xplayer))
\end{lstlisting}

\subsubsection{Derivation 1: Query Answering}
A sequence of resolution steps is called a \textbf{derivation}. A
\textbf{successful} derivation ends with the empty query ($\square$).

The \textbf{answer substitution} (computed by a successful derivation) is
obtained by composing the mgu's $\Theta_1 \circ ... \circ \Theta_n$ of each
step (and restricting the result to the variables in the original query).

A \textbf{failed} derivation ends with a query to which no clause applies.

\paragraph{Example}
\begin{lstlisting}
true(cell(1,1,b))
...
true(cell(3,3,b))
true(control(xplayer)

legal(P,mark(M,N)) $\Leftarrow$ true(cell(M,N,b)) $\land$ true(control(P))
legal(xplayer,noop) $\Leftarrow$ true(control(oplayer))
legal(oplayer,noop) $\Leftarrow$ true(control(xplayer))

legal(xplayer,M)
    $\rightarrow$ with $\Theta_1$ = {P5/xplayer,M/mark(M5,N5)}
true(cell(M5,N5,b)) $\land$ true(control(xplayer))
    $\rightarrow$ with $\Theta_2$ = {M5/1,N5/1}
true(control(xplayer))
    $\rightarrow$ with $\Theta_3$ = {}
$\square$, with answer substitution {M/mark(1,1)}
\end{lstlisting}

The query \verb|?- legal(xplayer,M)| has 9 successful derivations, with the
following answers:

\begin{lstlisting}
{M/mark(1,1)}
...
{M/mark(3,3)}
\end{lstlisting}

The query \verb|?- legal(oplayer,M)| has 1 successful derivation, with the
answer \verb|{M/noop}|.

\subsubsection{Derivation 2: Query Answering with Negation}
Given a query $L_1 \land L_2 \land ... \land L_m$ and clauses:
\begin{itemize}
    \item If $L_1$ is an atom, proceed as before
    \item If $L_1$ is of the form $\neg$A
        \begin{itemize}
            \item if all derivations for A fail then\\
                $\neg A \land L_2 \land ... \land L_m \rightarrow L_2 \land ...
                \land L_m$
            \item if there is a successful derivation for A then\\
                $\neg A \land L_2 \land ... \land L_m \rightarrow$ fail
        \end{itemize}
\end{itemize}

\paragraph{Example}
\begin{lstlisting}
role(red)
role(blue)
role(green)
true(freecell(blue))
trapped(P) $\Leftarrow$ role(P) $\land$ $\neg$true(freecell(P))
goal(P,100) $\Leftarrow$ role(P) $\land$ $\neg$trapped(P)

goal(P,100) $\rightarrow$ role(P) $\land$ $\neg$trapped(P) $\rightarrow$ $\neg$trapped(blue)
  and
    trapped(blue) $\rightarrow$ role(blue) $\land$ $\neg$true(freecell(blue))
  but we have true(freecell(blue)) so trapped(blue) fails

as such, we have goal(P,100) $\rightarrow$ role(P) $\land$ $\neg$[fails]
\end{lstlisting}

and so the answer substitution is \verb|{P/blue}|, which is the only answer to
this query.

\subsubsection{Derivation 3: Query Answering with Disjunction}
A GDL/Prolog rule with a disjunction\\
\indent A $\Leftarrow$ $B \land (C_1 \lor C_2) \land D$\\
is logically equivalent to the conjunction of the clauses\\
\indent A $\Leftarrow$ $B \land C_1 \land D$\\
\indent A $\Leftarrow$ $B \land C_2 \land D$\\

As such,\\
\indent line(W) $\Leftarrow$ row(M,W) $\lor$ column(N,W) $\lor$ diagonal(W)\\
is logically equivalent to
\begin{lstlisting}
line(W) $\Leftarrow$ row(M,W)
line(W) $\Leftarrow$ column(N,W)
line(W) $\Leftarrow$ diagonal(W)
\end{lstlisting}

\section[Logic in Action]{Logic in Action: Inference Tasks for General Game-Playing Agents}
Let KB be the logical description of a game.
\begin{enumerate}
    \item To infer the initial position, compute all answers to KB $\models$ init(F)
    \item To infer the legal moves of player $p$ in a given state
        $\{f_1,...,f_n\}$ compute all answers to KB $\land true(f_1) \land ...
        \land true(f_n) \models legal(p,M)$
    \item To infer the next state from a given state and moves, compute al
        answers to KB $\land true(f_1) \land ... \land true(f_n) \land
        does(p_1,f_1) \land ... \land does(p_n,f_n) \models$ next(F)
    \item To check if a state is terminal: KB $\land true(f_1) \land ... \land
        true(f_n) \models$ terminal
    \item To compute player $p$'s goal value: KB $\land true(f_1) \land ...
        \land true(f_n) \models$ goal(p,N)
\end{enumerate}
